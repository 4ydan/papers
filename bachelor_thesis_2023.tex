% Copyright (C) 2014-2023 by Thomas Auzinger <thomas@auzinger.name>

\documentclass[draft, final]{vutinfth} % Remove option 'final' to obtain debug information.

% Load packages to allow in- and output of non-ASCII characters.
\usepackage{lmodern}        % Use an extension of the original Computer Modern font to minimize the use of bitmapped letters.
\usepackage[T1]{fontenc}    % Determines font encoding of the output. Font packages have to be included before this line.
\usepackage[utf8]{inputenc} % Determines encoding of the input. All input files have to use UTF8 encoding.

% Extended LaTeX functionality is enables by including packages with \usepackage{...}.
\usepackage{amsmath}    % Extended typesetting of mathematical expression.
\usepackage{amssymb}    % Provides a multitude of mathematical symbols.
\usepackage{mathtools}  % Further extensions of mathematical typesetting.
\usepackage{microtype}  % Small-scale typographic enhancements.
\usepackage[inline]{enumitem} % User control over the layout of lists (itemize, enumerate, description).
\usepackage{multirow}   % Allows table elements to span several rows.
\usepackage{booktabs}   % Improves the typesetting of tables.
\usepackage{subcaption} % Allows the use of subfigures and enables their referencing.
\usepackage[ruled,linesnumbered,algochapter]{algorithm2e} % Enables the writing of pseudo code.
\usepackage[usenames,dvipsnames,table]{xcolor} % Allows the definition and use of colors. This package has to be included before tikz.
\usepackage{nag}       % Issues warnings when best practices in writing LaTeX documents are violated.
\usepackage{todonotes} % Provides tooltip-like todo notes.
\usepackage{hyperref}  % Enables hyperlinking in the electronic document version. This package has to be included second to last.
\usepackage[acronym,toc]{glossaries} % Enables the generation of glossaries and lists of acronyms. This package has to be included last.

% Custom packages
\usepackage{subfiles} % Best loaded last in the preamble
\usepackage{epigraph}
% \usepackage{hyperref}

% Define convenience functions to use the author name and the thesis title in the PDF document properties.
\newcommand{\authorname}{Aydan Namdar Ghazani} % The author name without titles.
\newcommand{\thesistitle}{Accelerating Software Integration and Delivery} % The title of the thesis. The English version should be used, if it exists.
\newcommand{\thesissubtitle}{A Case Study on a FLOSS Project} % The title of the thesis. The English version should be used, if it exists.

% Set PDF document properties
\hypersetup{
    pdfpagelayout   = TwoPageRight,           % How the document is shown in PDF viewers (optional).
    linkbordercolor = {Melon},                % The color of the borders of boxes around hyperlinks (optional).
    pdfauthor       = {\authorname},          % The author's name in the document properties (optional).
    pdftitle        = {\thesistitle},         % The document's title in the document properties (optional).
    pdfsubject      = {Subject},              % The document's subject in the document properties (optional).
    pdfkeywords     = {a, list, of, keywords} % The document's keywords in the document properties (optional).
}

\setpnumwidth{2.5em}        % Avoid overfull hboxes in the table of contents (see memoir manual).
\setsecnumdepth{subsection} % Enumerate subsections.
\setcounter{tocdepth}{1}

\nonzeroparskip             % Create space between paragraphs (optional).
\setlength{\parindent}{0pt} % Remove paragraph indentation (optional).

\makeindex      % Use an optional index.
\makeglossaries % Use an optional glossary.
%\glstocfalse   % Remove the glossaries from the table of contents.

% Set persons with 4 arguments:
%  {title before name}{name}{title after name}{gender}
%  where both titles are optional (i.e. can be given as empty brackets {}).
\setauthor{}{\authorname}{}{male}
\setadvisor{Univ.Lektor Dipl.-Ing. Dr.techn.}{Markus Raab}{BSc}{male}

% For bachelor and master theses:
% \setfirstassistant{Pretitle}{Forename Surname}{Posttitle}{male}
% \setsecondassistant{Pretitle}{Forename Surname}{Posttitle}{male}
% \setthirdassistant{Pretitle}{Forename Surname}{Posttitle}{male}

% For dissertations:
% \setfirstreviewer{Pretitle}{Forename Surname}{Posttitle}{male}
% \setsecondreviewer{Pretitle}{Forename Surname}{Posttitle}{male}

% For dissertations at the PhD School and optionally for dissertations:
% \setsecondadvisor{Pretitle}{Forename Surname}{Posttitle}{male} % Comment to remove.

% Required data.
\setregnumber{11709245}
\setdate{08}{07}{2023} % Set date with 3 arguments: {day}{month}{year}.
\settitle{\thesistitle}{}
\setsubtitle{\thesissubtitle}{}

% Select the thesis type: bachelor / master / doctor.
% Bachelor:
\setthesis{bachelor}
%
% Master:
%\setthesis{master}
%\setmasterdegree{dipl.} % dipl. / rer.nat. / rer.soc.oec. / master
%
% Doctor:
%\setthesis{doctor}
%\setdoctordegree{rer.soc.oec.}% rer.nat. / techn. / rer.soc.oec.

% For bachelor and master:
\setcurriculum{Software \& Information Engineering}{Software \& Information Engineering} % Sets the English and German name of the curriculum.

% Optional reviewer data:
\setfirstreviewerdata{Affiliation, Country}
\setsecondreviewerdata{Affiliation, Country}

\begin{document}
\epigraphfontsize{\small\itshape}
\setlength\epigraphwidth{14cm}
\setlength\epigraphrule{0pt}

\frontmatter % Switches to roman numbering.
% The structure of the thesis has to conform to the guidelines at
%  https://informatics.tuwien.ac.at/study-services

%\addtitlepage{naustrian} % German title page.
\addtitlepage{english} % English title page.
\addstatementpage

% \vspace*{\fill}
% \begin{center}
% To those who always stood behind me
% \end{center}
% \vfill

\begin{acknowledgements*}

I want to thank my family for always supporting me,

Markus and Yvonne for this wonderful opportunity of a project,

and everyone else who stood behind me, after all this time.

\end{acknowledgements*}

% ACRONYMS
\newacronym{ci}{CI}{Continuous Integration}
\newacronym{cd}{CD}{Continuous Delivery}
\newacronym{ct}{CT}{Continuous Testing}
\newacronym{cicd}{CI/CD}{Continuous Integration and Delivery}
\newacronym{scm}{SCM}{Source Code Management}
\newacronym{vcs}{VCS}{Version Control System}
\newacronym{e2e}{E2E}{End-to-End}
\newglossaryentry{FLOSS}
{
  name={FLOSS},
  description={An abbreviation for Free/Libre and Open Source Software, refers to software that is both free in terms of user freedom and open source in terms of its accessible source code \cite{FLOSS/FOSS}.}
}
\newglossaryentry{worker}
{
  name={worker},
  description={is a computing unit within a distributed environment that performs tasks or computations.}
}

\begin{kurzfassung}
    PermaplanT ist ein innovatives \gls{FLOSS} Projekt.
    Um den Entwicklungsprozess zu beschleunigen, sind gute \gls{cicd} Praktiken unerlässlich.
    Durch die Implementierung einer effizienten deployment pipeline können Entwickler schnell Feedback zu ihren Änderungen erhalten, frühzeitig Fehler erkennen und widerstandsfähigeren Code produzieren.
    Durch kontinuierliche Tests und Validierung wird sichergestellt, dass das Projekt wie beabsichtigt funktioniert und ein hohes Qualitätsniveau aufrechterhalten wird.
    Dieser iterative Ansatz erleichtert effiziente Feedback-Schleifen und ermöglicht es dem Projekt, sich schnell an die sich ändernde Benutzeranforderungen anzupassen.
    Die Erstellung einer nachhaltigen deployment pipeline mit automatisierten Tests hilft PermaplanT, eine benutzerzentrierte, anpassungsfähige Lösung mit möglichen häufigen Softwareupdates zu liefern.
\end{kurzfassung}

\begin{abstract}
    PermaplanT is a new cutting-edge \gls{FLOSS} project.
    To accelerate its development process, the establishment of well-structured Continuous Integration and Continuous Delivery practices are essential.
    By implementing an efficient deployment pipeline, developers can get rapid feedback on their changes, detect bugs early and produce more resilient code.
    Continuously testing and validating each iteration, ensures that the project functions as intended, and maintains a high level of quality.
    This iterative approach facilitates efficient feedback loops, enabling the project to quickly adapt to changing user requirements.
    The creation of a sustainable deployment pipeline with automated tests helps PermaplanT deliver a user-centric, adaptable solution with  frequent software updates.
\end{abstract}

% Select the language of the thesis, e.g., english or naustrian.
\selectlanguage{english}

% Add a table of contents (toc).
\tableofcontents % Starred version, i.e., \tableofcontents*, removes the self-entry.

% Switch to arabic numbering and start the enumeration of chapters in the table of content.
\mainmatter

% ALL CHAPTERS
\subfile{introduction}
\subfile{terminology}
\subfile{problem}
\subfile{method}
\subfile{results}
\subfile{conclusion}

\backmatter

% Use an optional list of figures.
\listoffigures % Starred version, i.e., \listoffigures*, removes the toc entry.

% Use an optional list of tables.
\cleardoublepage % Start list of tables on the next empty right hand page.
\listoftables % Starred version, i.e., \listoftables*, removes the toc entry.

% Use an optional list of algorithms.
% \listofalgorithms
% \addcontentsline{toc}{chapter}{List of Algorithms}

% Add an index.
\printindex

% Add a glossary.
\printglossaries

% Add a bibliography.
\bibliographystyle{ieeetr}
\bibliography{bib}

\end{document}
